%%%%%%%%%%%%%%%%%%%%%%%%%%%%%%%%%%%%%%%%%
% Beamer Presentation
% LaTeX Template
% Version 2.0 (March 8, 2022)
%
% This template originates from:
% https://www.LaTeXTemplates.com
%
% Author:
% Vel (vel@latextemplates.com)
%
% License:
% CC BY-NC-SA 4.0 (https://creativecommons.org/licenses/by-nc-sa/4.0/)
%
%%%%%%%%%%%%%%%%%%%%%%%%%%%%%%%%%%%%%%%%%

%%%%%%%%%%%%%%%%%%%%%%%%%%%%%%%%%%%%%%%%%
% This presentation template is an adaptation of the template mentioned above. It has been created by Giovanni Spadaro and it is available on GitHub (https://github.com/Giovo17/presentation-template-unict-lm-data).
%%%%%%%%%%%%%%%%%%%%%%%%%%%%%%%%%%%%%%%%%
%----------------------------------------------------------------------------------------
% 	MODIFY FOR SON NGUYEN THESIS. Available on Github (https://github.com/sonnh-uit/HCMUIT_thesistemplate/tree/master/presentation)
%----------------------------------------------------------------------------------------
%	PACKAGES AND OTHER DOCUMENT CONFIGURATIONS
%----------------------------------------------------------------------------------------

\documentclass[
	11pt, % Set the default font size, options include: 8pt, 9pt, 10pt, 11pt, 12pt, 14pt, 17pt, 20pt
	t, % Uncomment to vertically align all slide content to the top of the slide, rather than the default centered
	aspectratio=169, % Uncomment to set the aspect ratio to a 16:9 ratio which matches the aspect ratio of 1080p and 4K screens and projectors
]{beamer}

\graphicspath{{img/}} % Specifies where to look for included images (trailing slash required)

\usepackage{booktabs} % Allows the use of \toprule, \midrule and \bottomrule for better rules in tables

%----------------------------------------------------------------------------------------
%	SELECT LAYOUT THEME
%----------------------------------------------------------------------------------------

% Beamer comes with a number of default layout themes which change the colors and layouts of slides. Below is a list of all themes available, uncomment each in turn to see what they look like.

%\usetheme{default}
%\usetheme{AnnArbor}
%\usetheme{Antibes}
%\usetheme{Bergen}
%\usetheme{Berkeley}
%\usetheme{Berlin}
% \usetheme{Boadilla} This 
%\usetheme{CambridgeUS}
%\usetheme{Copenhagen}
%\usetheme{Darmstadt}
%\usetheme{Dresden}
%\usetheme{Frankfurt}
%\usetheme{Goettingen}
%\usetheme{Hannover}
%\usetheme{Ilmenau}
%\usetheme{JuanLesPins}
%\usetheme{Luebeck}
\usetheme{Madrid}
%\usetheme{Malmoe}
%\usetheme{Marburg}
%\usetheme{Montpellier}
%\usetheme{PaloAlto}
%\usetheme{Pittsburgh}
%\usetheme{Rochester}
%\usetheme{Singapore}
%\usetheme{Szeged}
%\usetheme{Warsaw}

%----------------------------------------------------------------------------------------
%	SELECT COLOR THEME
%----------------------------------------------------------------------------------------

% Beamer comes with a number of color themes that can be applied to any layout theme to change its colors. Uncomment each of these in turn to see how they change the colors of your selected layout theme.

%\usecolortheme{albatross}
%\usecolortheme{beaver}   % red
%\usecolortheme{beetle}
%\usecolortheme{crane}   % yellow
%\usecolortheme{dolphin}  % purple
%\usecolortheme{dove}   % white
%\usecolortheme{fly}   % grey
%\usecolortheme{lily}   % purple
%\usecolortheme{monarca}   % yellow background and black
%\usecolortheme{seagull}
%\usecolortheme{seahorse}
%\usecolortheme{spruce}   % green
\usecolortheme{whale}
%\usecolortheme{wolverine}

%	CHANGE COLOR FOLLOW BY TEMPLATE OF Lucas Amaral Taylor. Available on Overleaf (https://www.overleaf.com/latex/templates/template-de-apresentacao-ime-usp/rygbgpftsdbk)
\definecolor{primaryColor}{RGB}{20,45,105} 
\definecolor{secondaryColor}{RGB}{0,100,160} 
\setbeamercolor{structure}{fg=primaryColor}
\setbeamercolor{palette primary}{bg=primaryColor, fg=white}
\setbeamercolor{palette secondary}{bg=secondaryColor, fg=white}
\setbeamercolor{title}{bg=primaryColor, fg=white} 
\setbeamercolor{headline}{bg=secondaryColor, fg=white}
\setbeamercolor{section in head/foot}{bg=primaryColor, fg=white}
\setbeamercolor{subsection in head/foot}{bg=secondaryColor, fg=white}
\setbeamercolor{author in head/foot}{bg=primaryColor, fg=white} 
\setbeamercolor{title in head/foot}{bg=secondaryColor, fg=white} 
\setbeamercolor{date in head/foot}{bg=primaryColor, fg=white} 
\setbeamercolor{page number in head/foot}{bg=primaryColor, fg=white} 

%----------------------------------------------------------------------------------------
%	SELECT FONT THEME & FONTS
%----------------------------------------------------------------------------------------

% Beamer comes with several font themes to easily change the fonts used in various parts of the presentation. Review the comments beside each one to decide if you would like to use it. Note that additional options can be specified for several of these font themes, consult the beamer documentation for more information.

% \usefonttheme{default} % Typeset using the default sans serif font
\usefonttheme{serif} % Typeset using the default serif font (make sure a sans font isn't being set as the default font if you use this option!)
%\usefonttheme{structurebold} % Typeset important structure text (titles, headlines, footlines, sidebar, etc) in bold
%\usefonttheme{structureitalicserif} % Typeset important structure text (titles, headlines, footlines, sidebar, etc) in italic serif
%\usefonttheme{structuresmallcapsserif} % Typeset important structure text (titles, headlines, footlines, sidebar, etc) in small caps serif

%----------------------------------------------------------------------------------------
%	USE PACKAGE 
%----------------------------------------------------------------------------------------
%\usepackage{mathptmx} % Use the Times font for serif text
\usepackage{palatino} % Use the Palatino font for serif text
%\usepackage{helvet} % Use the Helvetica font for sans serif text
\usepackage[default]{opensans} % Use the Open Sans font for sans serif text
%\usepackage[default]{FiraSans} % Use the Fira Sans font for sans serif text
%\usepackage[default]{lato} % Use the Lato font for sans serif text
\usepackage{subcaption}
\usepackage[alf]{abntex2cite}
\usepackage[utf8]{vietnam}
%----------------------------------------------------------------------------------------
%	PACOTES E CONFIGURAÇÕES PARA CÓDIGO
%----------------------------------------------------------------------------------------
% Pacotes necessários para formatação de código
\usepackage[utf8]{inputenc}
\usepackage{listings}
\usepackage{xcolor}

% Cores para syntax highlighting (VSCode Light Theme)
\definecolor{vscBackground}{RGB}{255,255,255}    % Fundo branco
\definecolor{vscKeyword}{RGB}{175,0,219}         % Roxo para palavras-chave
\definecolor{vscString}{RGB}{163,21,21}          % Vermelho para strings
\definecolor{vscComment}{RGB}{0,128,0}           % Verde para comentários
\definecolor{vscFunction}{RGB}{121,94,38}        % Marrom para funções
\definecolor{vscNumber}{RGB}{9,134,88}           % Verde escuro para números
\definecolor{vscOperator}{RGB}{175,0,219}        % Roxo para operadores
\definecolor{vscText}{RGB}{0,0,0}                % Texto preto
\definecolor{vscLineNr}{RGB}{128,128,128}        % Cinza para números de linha

% Configuração geral do listings para UTF-8
\lstset{
    inputencoding=utf8,
    extendedchars=true,
    literate=%
        {á}{{\'a}}1 {é}{{\'e}}1 {í}{{\'i}}1 {ó}{{\'o}}1 {ú}{{\'u}}1
        {Á}{{\'A}}1 {É}{{\'E}}1 {Í}{{\'I}}1 {Ó}{{\'O}}1 {Ú}{{\'U}}1
        {à}{{\`a}}1 {è}{{\`e}}1 {ì}{{\`i}}1 {ò}{{\`o}}1 {ù}{{\`u}}1
        {À}{{\`A}}1 {È}{{\'E}}1 {Ì}{{\`I}}1 {Ò}{{\`O}}1 {Ù}{{\`U}}1
        {ã}{{\~a}}1 {ẽ}{{\~e}}1 {ĩ}{{\~i}}1 {õ}{{\~o}}1 {ũ}{{\~u}}1
        {Ã}{{\~A}}1 {Ẽ}{{\~E}}1 {Ĩ}{{\~I}}1 {Õ}{{\~O}}1 {Ũ}{{\~U}}1
        {â}{{\^a}}1 {ê}{{\^e}}1 {î}{{\^i}}1 {ô}{{\^o}}1 {û}{{\^u}}1
        {Â}{{\^A}}1 {Ê}{{\^E}}1 {Î}{{\^I}}1 {Ô}{{\^O}}1 {Û}{{\^U}}1
        {ç}{{\c c}}1 {Ç}{{\c C}}1
        {º}{{\textordmasculine}}1
        {ª}{{\textordfeminine}}1
}

% Configurações base comum para todas as linguagens
\lstdefinestyle{baseStyle}{
    backgroundcolor=\color{vscBackground},
    basicstyle=\ttfamily\small\color{vscText},
    breakatwhitespace=false,
    breaklines=true,
    captionpos=b,
    keepspaces=true,
    numbers=left,
    numbersep=5pt,
    showspaces=false,
    showstringspaces=false,
    showtabs=false,
    tabsize=4,
    frame=single,
    framerule=0.8pt,
    rulecolor=\color{gray!20},
    numberstyle=\tiny\color{vscLineNr},
    keywordstyle=\color{vscKeyword},
    commentstyle=\color{vscComment}\itshape,
    stringstyle=\color{vscString},
    emphstyle=\color{vscFunction},
    columns=flexible,
    basewidth={0.5em,0.45em},
    inputencoding=utf8,
    extendedchars=true
}

%----------------------------------------------------------------------------------------
% Python
%----------------------------------------------------------------------------------------
\lstdefinestyle{pythonStyle}{
    style=baseStyle,
    language=Python,
    morekeywords={self,None,True,False,import,from,as,def,class,return,yield,
                  for,while,if,else,elif,try,except,finally,with,lambda,
                  async,await,break,continue,global,nonlocal,pass,raise},
    morekeywords=[2]{print,len,range,type,int,str,float,list,dict,set,
                     tuple,max,min,sum,sorted,enumerate,zip,map,filter,
                     any,all,abs,round,pow,divmod},
    keywordstyle=[2]\color{vscFunction},
    sensitive=true
}

\lstnewenvironment{python}[1][]{\lstset{style=pythonStyle, #1}}{}
\newcommand{\pyinline}[1]{\lstinline[style=pythonStyle]!#1!}
\newcommand{\inputpython}[2][]{\lstinputlisting[style=pythonStyle,#1]{#2}}

%----------------------------------------------------------------------------------------
% C Language
%----------------------------------------------------------------------------------------
\lstdefinestyle{cStyle}{
    style=baseStyle,
    language=C,
    morekeywords={include,define,void,int,char,float,double,long,unsigned,
                  struct,union,enum,typedef,const,static,extern,register,
                  auto,volatile,sizeof,return,if,else,for,while,do,switch,
                  case,break,continue,default,goto},
    morekeywords=[2]{printf,scanf,malloc,free,calloc,realloc,fopen,fclose,
                     fprintf,fscanf,strcpy,strlen,strcat},
    keywordstyle=[2]\color{vscFunction},
    sensitive=true
}

\lstnewenvironment{clang}[1][]{\lstset{style=cStyle, #1}}{}
\newcommand{\clinline}[1]{\lstinline[style=cStyle]!#1!}
\newcommand{\inputclang}[2][]{\lstinputlisting[style=cStyle,#1]{#2}}

%----------------------------------------------------------------------------------------
% C++
%----------------------------------------------------------------------------------------
\lstdefinestyle{cppStyle}{
    style=baseStyle,
    language=C++,
    morekeywords={class,private,protected,public,template,typename,namespace,
                  using,new,delete,this,friend,virtual,override,final,explicit,
                  mutable,constexpr,nullptr,noexcept,static_cast,dynamic_cast,
                  const_cast},
    morekeywords=[2]{cout,cin,endl,vector,string,map,set,queue,stack,pair,
                     begin,end,push_back,pop_back,emplace_back,size,empty},
    keywordstyle=[2]\color{vscFunction},
    sensitive=true
}

\lstnewenvironment{cpp}[1][]{\lstset{style=cppStyle, #1}}{}
\newcommand{\cppinline}[1]{\lstinline[style=cppStyle]!#1!}
\newcommand{\inputcpp}[2][]{\lstinputlisting[style=cppStyle,#1]{#2}}

%----------------------------------------------------------------------------------------
% R Language
%----------------------------------------------------------------------------------------
\lstdefinestyle{rStyle}{
    style=baseStyle,
    language=R,
    morekeywords={if,else,repeat,while,function,for,in,next,break,TRUE,FALSE,
                  NULL,Inf,NaN,NA,NA_integer_,NA_real_,NA_complex_,NA_character_},
    morekeywords=[2]{library,require,attach,detach,source,setwd,options,
                     data.frame,read.csv,write.csv,list,matrix,array},
    keywordstyle=[2]\color{vscFunction},
    sensitive=true
}

\lstnewenvironment{rlang}[1][]{\lstset{style=rStyle, #1}}{}
\newcommand{\rlinline}[1]{\lstinline[style=rStyle]!#1!}
\newcommand{\inputrlang}[2][]{\lstinputlisting[style=rStyle,#1]{#2}}

%----------------------------------------------------------------------------------------
% Java
%----------------------------------------------------------------------------------------
\lstdefinestyle{javaStyle}{
    style=baseStyle,
    language=Java,
    morekeywords={abstract,assert,boolean,break,byte,case,catch,char,class,
                  const,continue,default,do,double,else,enum,extends,final,
                  finally,float,for,if,implements,import,instanceof,int,
                  interface,long,native,new,package,private,protected,public,
                  return,short,static,strictfp,super,switch,synchronized,this,
                  throw,throws,transient,try,void,volatile,while},
    morekeywords=[2]{String,System,out,println,printStackTrace,ArrayList,
                     HashMap,Arrays,List,Map,Set,Exception,RuntimeException},
    keywordstyle=[2]\color{vscFunction},
    sensitive=true
}

\lstnewenvironment{java}[1][]{\lstset{style=javaStyle, #1}}{}
\newcommand{\javainline}[1]{\lstinline[style=javaStyle]!#1!}
\newcommand{\inputjava}[2][]{\lstinputlisting[style=javaStyle,#1]{#2}}
%----------------------------------------------------------------------------------------
%	SELECT INNER THEME
%----------------------------------------------------------------------------------------

% Inner themes change the styling of internal slide elements, for example: bullet points, blocks, bibliography entries, title pages, theorems, etc. Uncomment each theme in turn to see what changes it makes to your presentation.

%\useinnertheme{default}
\useinnertheme{circles}
%\useinnertheme{rectangles}
%\useinnertheme{rounded}
%\useinnertheme{inmargin}

%----------------------------------------------------------------------------------------
%	SELECT OUTER THEME
%----------------------------------------------------------------------------------------

% Outer themes change the overall layout of slides, such as: header and footer lines, sidebars and slide titles. Uncomment each theme in turn to see what changes it makes to your presentation.

%\useoutertheme{default}
%\useoutertheme{infolines}
\useoutertheme{miniframes}
%\useoutertheme{smoothbars}
%\useoutertheme{sidebar}
%\useoutertheme{split}
%\useoutertheme{shadow}
%\useoutertheme{tree}
%\useoutertheme{smoothtree}

%\setbeamertemplate{footline} % Uncomment this line to remove the footer line in all slides
%\setbeamertemplate{footline}[page number] % Uncomment this line to replace the footer line in all slides with a simple slide count

%\setbeamertemplate{navigation symbols}{} % Uncomment this line to remove the navigation symbols from the bottom of all slides

%----------------------------------------------------------------------------------------
%	PRESENTATION INFORMATION
%----------------------------------------------------------------------------------------

\title[Short title]{Một tiêu đề luận văn bằng tiếng Việt thật là dài} % The short title in the optional parameter appears at the bottom of every slide, the full title in the main parameter is only on the title page

\subtitle{This is an English title} % Presentation subtitle, remove this command if a subtitle isn't required

\author[Son Nguyen]{Nguyen Hong Son} % Presenter name(s), the optional parameter can contain a shortened version to appear on the bottom of every slide, while the main parameter will appear on the title slide

\institute[UIT]{Instructor: Dr Le Duy Tan} % Your institution, the optional parameter can be used for the institution shorthand and will appear on the bottom of every slide after author names, while the required parameter is used on the title slide and can include your email address or additional information on separate lines

\date[\today]{\today} % Presentation date or conference/meeting name, the optional parameter can contain a shortened version to appear on the bottom of every slide, while the required parameter value is output to the title slide
% \logo{\includegraphics[scale=0.19]{img/nc-logo.png}}
%----------------------------------------------------------------------------------------
\AtBeginSection[]
{
  \begin{frame}
    \frametitle{Table of Contents}
    \tableofcontents[currentsection]
  \end{frame}
}

\begin{document}

%----------------------------------------------------------------------------------------
%	TITLE SLIDE
%----------------------------------------------------------------------------------------

\begin{frame}

    \begin{figure}
        \centering
		\includegraphics[scale=0.15]{img/nc-logo.png}
	\end{figure}
 
	\titlepage % Output the title slide, automatically created using the text entered in the PRESENTATION INFORMATION block above
\end{frame}

%----------------------------------------------------------------------------------------
%	TABLE OF CONTENTS SLIDE
%----------------------------------------------------------------------------------------

% The table of contents outputs the sections and subsections that appear in your presentation, specified with the standard \section and \subsection commands. You may either display all sections and subsections on one slide with \tableofcontents, or display each section at a time on subsequent slides with \tableofcontents[pausesections]. The latter is useful if you want to step through each section and mention what you will discuss.

\begin{frame}
	\frametitle{Table of Contents} % Slide title, remove this command for no title
	
	\tableofcontents % Output the table of contents (all sections on one slide)
	%\tableofcontents[pausesections] % Output the table of contents (break sections up across separate slides)
\end{frame}

%----------------------------------------------------------------------------------------
%	PRESENTATION BODY SLIDES
%----------------------------------------------------------------------------------------



\section{Create a slide with only text} % Sections are added in order to organize your presentation into discrete blocks, all sections and subsections are automatically output to the table of contents as an overview of the talk but NOT output in the presentation as separate slides

%------------------------------------------------

\begin{frame}
	\frametitle{Text paragraph}
    Lorem ipsum dolor sit amet, consectetur adipiscing elit. Nullam ipsum velit, cursus quis ligula eu, malesuada aliquet massa. Quisque non convallis felis, a auctor eros. Etiam sit amet turpis a sapien pulvinar malesuada quis quis nisi. Quisque scelerisque volutpat ligula vel mollis. Nam sit amet tristique erat, sit amet cursus mi. 
\end{frame}

%------------------------------------------------

\begin{frame}
	\frametitle{Text with enumerate }
     Lorem ipsum dolor sit amet, consectetur adipiscing elit:
    \begin{enumerate}
        \item Lorem ipsum dolor sit amet.
        \item Lorem ipsum dolor sit amet.
    \end{enumerate}
	
\end{frame}

%------------------------------------------------

\begin{frame}
	\frametitle{Text with itemize}
     Lorem ipsum dolor sit amet, consectetur adipiscing elit:
    \begin{itemize}
        \item Lorem ipsum dolor sit amet.
        \item Lorem ipsum dolor sit amet.
    \end{itemize}
	
\end{frame}

%------------------------------------------------



\section{Slide with images}
%------------------------------------------------

\begin{frame}{Single images}
   \begin{figure}[h]
       \centering
       \includegraphics[width=0.7\textwidth]{img/logo_v2.png}
       \caption{Slide with single images}
       \label{fig:sing_image}
   \end{figure}
\end{frame}

%------------------------------------------------

\begin{frame}{Single image with itemize}
     Lorem ipsum dolor sit amet, consectetur adipiscing elit:
    \begin{enumerate}
        \item Lorem ipsum dolor sit amet.
        \item Lorem ipsum dolor sit amet.
    \end{enumerate}
    
   \begin{figure}[h]
       \centering
       \includegraphics[width=0.7\textwidth]{img/logo_v2.png}
       \caption{Slide with single images}
       \label{fig:sing_image}
   \end{figure}
\end{frame}

%------------------------------------------------
\begin{frame}{Double images}
     Lorem ipsum dolor sit amet, consectetur adipiscing elit:
    \begin{enumerate}
        \item Lorem ipsum dolor sit amet.
        \item Lorem ipsum dolor sit amet.
    \end{enumerate}
\begin{figure}
    \begin{subfigure}{0.3\textwidth}
        \centering
        \includegraphics[scale=0.3]{img/logo.png}
        \caption{Caption of figure 1}
        \label{fig:sub-figure-url-1}
    \end{subfigure}
    \begin{subfigure}{0.3\textwidth}
        \centering
        \includegraphics[scale=0.3]{img/logo.png}
        \caption{Caption of figure 2}
        \label{fig:sub-figure-url-2}
    \end{subfigure}
    \caption{Caption of figure}
    \label{fig:example-subfigure}
\end{figure}
\end{frame}

%------------------------------------------------

\section{Equation and Code}

\begin{frame}{Equation}
    Navier-Stokes Equations Expanded Form (3D):
    \footnotesize
        \begin{align*}
            \rho\left(\frac{\partial u}{\partial t} + u\frac{\partial u}{\partial x} + v\frac{\partial u}{\partial y} + w\frac{\partial u}{\partial z}\right) &= -\frac{\partial p}{\partial x} + \mu\left(\frac{\partial^2 u}{\partial x^2} + \frac{\partial^2 u}{\partial y^2} + \frac{\partial^2 u}{\partial z^2}\right) + f_x \\[0.3cm]
            \rho\left(\frac{\partial v}{\partial t} + u\frac{\partial v}{\partial x} + v\frac{\partial v}{\partial y} + w\frac{\partial v}{\partial z}\right) &= -\frac{\partial p}{\partial y} + \mu\left(\frac{\partial^2 v}{\partial x^2} + \frac{\partial^2 v}{\partial y^2} + \frac{\partial^2 v}{\partial z^2}\right) + f_y \\[0.3cm]
            \rho\left(\frac{\partial w}{\partial t} + u\frac{\partial w}{\partial x} + v\frac{\partial w}{\partial y} + w\frac{\partial w}{\partial z}\right) &= -\frac{\partial p}{\partial z} + \mu\left(\frac{\partial^2 w}{\partial x^2} + \frac{\partial^2 w}{\partial y^2} + \frac{\partial^2 w}{\partial z^2}\right) + f_z
        \end{align*}
        
    where $\mathbf{v} = (u,v,w)$ is the velocity field, $p$ is the pressure, $\rho$ is the density, $\mu$ is the dynamic viscosity, and $\mathbf{f}$ represents external forces.
\end{frame}

%------------------------------------------------

\begin{frame}[fragile]
    \frametitle{Python}
    
    \begin{python}
def calcular_dobro(x):
    """Retorna o dobro do número"""
    return 2 * x

# Testando a função
numero = 5
resultado = calcular_dobro(numero)
print(f"O dobro de {numero} é {resultado}")
    \end{python}
\end{frame}

%------------------------------------------------
\begin{frame}[fragile]
    \frametitle{Java}
    
    \begin{java}
public class Exemplo {
    public static void main(String[] args) {
        int numero = 5;
        int dobro = 2 * numero;
        
        System.out.println("O dobro de " + numero +
                         " eh " + dobro);
    }
}
    \end{java}
\end{frame}
%------------------------------------------------

\section{Specific feature}
%------------------------------------------------
\begin{frame}
\frametitle{Slide with highligh text}

In this slide, some important text will be
\alert{highlighted} because it's important.
Please, don't abuse it.

\begin{block}{Remark}
Sample text
\end{block}

\begin{alertblock}{Important theorem}
Sample text in red box
\end{alertblock}

\begin{examples}
Sample text in green box. The title of the block is ``Examples".
\end{examples}
\end{frame}
%------------------------------------------------
\begin{frame}
\frametitle{Slide with transition}
In this slide \pause

the text will be partially visible \pause

And finally everything will be there
\end{frame}

%------------------------------------------------

\begin{frame}
\frametitle{Two-column slide}

\begin{columns}

\column{0.5\textwidth}
This is a text in first column.
$$E=mc^2$$
\begin{itemize}
\item First item
\item Second item
\end{itemize}

\column{0.5\textwidth}
This text will be in the second column
and on a second tought this is a nice looking
layout in some cases.
\end{columns}
\end{frame}

%------------------------------------------------
\section{Conclusion}

\begin{frame}{References}
    \footnotesize
    \begin{thebibliography}{99}
        \bibitem{MartinRotate} SCHARRER, Martin, \textit{Rotate picture with caption}, 2012. 
        Available at: \url{https://tex.stackexchange.com/questions/44427/rotate-picture-with-caption/57531\#57531}. 
        Accessed: April 8, 2024.
    \end{thebibliography}
\end{frame}



%----------------------------------------------------------------------------------------
%	CLOSING SLIDE
%----------------------------------------------------------------------------------------

\begin{frame}[plain] % The optional argument 'plain' hides the headline and footline
	\begin{center}
		{\Huge Thanks for your attention}
            
	\end{center}
\end{frame}

%----------------------------------------------------------------------------------------

\end{document} 