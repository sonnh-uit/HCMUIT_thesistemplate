% \chapter*{Tóm tắt khóa luận}
\chapter*{\centering\Large{Tóm tắt khóa luận}}
\addcontentsline{toc}{chapter}{Tóm tắt khóa luận}

Vào năm 2021, dù cảm thấy không có đủ năng lực để theo tiếp đồ án chuyên ngành, mình thực hiện khóa luận do chương trình đào tạo cũ bắt buộc làm luận văn. Nhóm mình hai người, thực hiện tìm hiểu và viết luận văn trong khoảng thời gian dịch bệnh đầy khó khăn.

Làm xong rồi, nhưng việc báo cáo nó cũng không phải dễ. Mỗi người một nơi, cùng làm việc trên Word khó khăn do định dạng không đồng bộ, chèn hình và bảng làm xáo trộn các trang, đánh số bảng biểu, hình ảnh, tiêu đề dễ bị sai sót. Khó khăn nhất là việc viết và đánh số tài liệu tham khảo.

Sau này, khi học thạc sĩ và làm việc trên linux, một mặt cần tính chuyên nghiệp cao trong viết báo cáo, mặt khác sử dụng Word online trên linux rất bất tiện, mình đã làm quen và dần dần chuyển hẳn việc viết báo cáo trên latex, kể cả các báo cáo ở công ty.

Nhằm lưu lại các cú pháp mình sử dụng, đồng thời giúp các bạn sinh viên viết khóa luận dễ dàng hơn, mình đã thiết kế và viết báo cáo này. Các bạn sinh viên UIT nói riêng và người dùng khác nói chung có thể tùy ý sử dụng, thay đổi dự án này tùy theo nhu cầu.

Phần còn lại của luận văn có cấu trúc như sau
\begin{itemize}
    \item Chương \ref{chap:chap1-introduce} giới thiệu khái quát chung về cấu trúc thư mục trong dự án, các cài đặt, cấu hình cần thay đổi khi sử dụng.
    \item Chương \ref{chap:chap2-figures} trình bày các kỹ thuật chèn hình ảnh.
    \item Chương \ref{chap:chap3-table} trình bày các kỹ thuật chèn bảng.
    \item Chương \ref{chap:chap4-summary} lưu ý một số các kỹ thuật còn lại.
\end{itemize}